\documentclass{beamer}

\usepackage{textcomp}       % Text companion fonts

\setbeamertemplate{navigation symbols}{}
\setbeamertemplate{items}[circle]

\hyphenpenalty 4000 \sloppy



\begin{document}


% ------------------------------------------------------------------------------
%
\begin{frame}{Improved Capabilities of new LMBF/TMBF}

The new LMBF is currently installed in the machine as \texttt{SR23C-DI-LMBF-01}
and provides the following improved capabilities:
\begin{itemize}
\item
    Much faster readout of bunch-by-bunch data: can read 100,000 turns in 1.6
    seconds, can capture up to 500,000 turns and readout in 8 seconds.
\item
    Can now measure separate tune responses for up to four different bunch
    patterns.
\end{itemize}

All functionality from the old TMBF is available except for the following which
will be implemented shortly:
\begin{itemize}
\item
    Automated setup scripts
\item
    Tune fitting to measured tune
\item
    Tune PLL
\end{itemize}

\end{frame}

% ------------------------------------------------------------------------------
%
\begin{frame}{Future Capabilities of new LMBF/TMBF}

First we need to replace the Libera TMBF with the new system.

\bigskip

Then, the following ideas for future capabilities are being considered:

\begin{itemize}
\item
    Filtering out mode zero
\item
    Selective control of one or more selected individual modes
\item
    Simultaneous tune sweeping with multiple excitation frequencies
\item
    Tune PLL tracking of grow-damp experiments
\end{itemize}

There remains room in the firmware for numerous future ideas.

\end{frame}

\end{document}
